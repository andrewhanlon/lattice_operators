\documentclass[12pt]{article}
\usepackage{graphicx}
\usepackage[letterpaper,text={6.5in,9in},centering]{geometry}
\usepackage{bm}
\usepackage{verbatim}
\usepackage{color}
%\usepackage{amsmath}
\usepackage{amsfonts}
\usepackage{fancyvrb}
\usepackage[hidelinks]{hyperref}

\usepackage{listings}
\usepackage{color}

\definecolor{dkgreen}{rgb}{0,0.6,0}
\definecolor{gray}{rgb}{0.5,0.5,0.5}
\definecolor{mauve}{rgb}{0.58,0,0.82}

\lstset{frame=tb,
  language=Python,
  aboveskip=3mm,
  belowskip=3mm,
  showstringspaces=false,
  columns=flexible,
  basicstyle={\small\ttfamily},
  numbers=none,
  numberstyle=\tiny\color{gray},
  keywordstyle=\color{blue},
  commentstyle=\color{dkgreen},
  stringstyle=\color{mauve},
  breaklines=true,
  breakatwhitespace=true,
  tabsize=4
}

%\setlength{\topmargin}{0pt}
%\setlength{\textheight}{9true in}
%\setlength{\oddsidemargin}{0 true in}
%\setlength{\textwidth}{6.5 true in}
\setlength{\unitlength}{1.25 true mm}
\setlength{\fboxsep}{5 true mm}
\renewcommand{\arraystretch}{1.4}
\newcommand{\sigmond}{\texttt{sigmond} }
\newcommand{\vb}{\texttt}
\renewcommand\labelitemi{--}

\begin{document}
\title{\bf Operators Python Package}
\author{Andrew D. Hanlon\\
  Helmholtz-Institut Mainz\\
  \texttt{ahanlon@uni-mainz.de}}
\date{\today}
\maketitle

%\newpage

\section{Quick Start}

To help get started, let us go over a simple example that constructs a set of Delta like Baryon operator
with momentum $\bm{P} = (0,0,1)$ and then outputs the Little group irrep decomposition.
We start by first importing the operator module and the sympy package

\begin{lstlisting}
  from sympy import *
  from operators.operators import *
\end{lstlisting}

Next, we need to construct some quark fields that will be used to construct the operator. Since our
Delta operator will be constructed from quark fields of flavor 'u', we construct these as

\begin{lstlisting}
  u = QuarkField.create('u')
\end{lstlisting}

Note that the operator knows nothing of quark flavor, and the 'u' passed to create is just a name.
Next, we need to create some indices

\begin{lstlisting}
  a = ColorIdx('a')
  b = ColorIdx('b')
  c = ColorIdx('c')

  i = DiracIdx('i')
  j = DiracIdx('j')
  k = DiracIdx('k')
\end{lstlisting}

Then, we will construct a set of Delta operators (one for each spin index as follows)

\begin{lstlisting}
  Delta = Eijk(a,b,c) * u[a,i] * u[b,j] * u[c,k]
\end{lstlisting}

Note that the color index comes before the Dirac spin index in a quark field.
Finally, we would like to put each of the operators into a list as follows

\begin{lstlisting}
ops = list()
for i_int in range(4):
  for j_int in range(i_int,4):
    for k_int in range(j_int,4):
      op = delta.subs({i:i_int,j:j_int,k:k_int})
      ops.append(Operator(op, Momentum([0,0,1])))
\end{lstlisting}

Then, the list of operators represents a possible operator basis, which we pass into
the class OperatorRepresentation, which will determin the irrep decomposition

\begin{lstlisting}
op_rep = OperatorRepresentation(*ops)

print("\nMoving Delta:")
print(op_rep.littleGroupContents(True, False))
\end{lstlisting}

The output of this code is

\begin{lstlisting}
6 G1 + 4 G2
\end{lstlisting}

which tells us that this basis contains 6 copies of the $G_1$ irrep and 4 copies of the $G_2$irrep.

\end{document}
